\section{Experiments}
In this section, we evaluate some of the existing state-of-the-art methods for multi-label OOD detection
 on public datasets in order to assess how they compare on identical settings. 
 To compare state-of-the-art methods, we will be using pre-trained networks with identical weights and experimentation methodology.

To provide a broad overview of the methods we select state-of-the-art method from each respective family.
\textbf{JointEnergy} as an instance of output-based, \textbf{Mahalanobis} for feature-based and \textbf{GradNorm} for gradient-based detectors.
All three methods were implemented to have the same evaluation pipeline. 

The evaluation was performed on datasets that have been utilized across multiple papers about multi-label OOD detection. 
Consequently, two key outcomes will be obtained:
\begin{itemize}
    \itemsep0em 
    \item new comparison of the methods because we used different datasets
    \item evaluation of GradNorm method in multi-label settings which has not been done yet
\end{itemize}
For the purpose of evaluation, datasets utilized in both of these works were utilized. 
The labels of the ID datasets are different from the labels of the OOD datasets in this evaluation to reduce bias.

\subsection{Experimental setup}
\paragraph{ID datasets}
We will use two multi-label datasets: MS-COCO\cite{linMicrosoftCOCOCommon2015} and PASCAL-VOC\cite{everinghamPascalVisualObject2015}. 
These datasets are used to train multi-label classifier and for producing ID reliability scores. 

\paragraph{OOD datasets}
To assess the models trained on the ID datasets, we adopt the same approach as described in \cite{Wang2021}. 
We select a subset of ImageNet\cite{dengImageNetLargescaleHierarchical2009} dataset, the whole Textures\cite{cimpoiDescribingTexturesWild2013} dataset and the whole TACO\cite{proencaTACOTrashAnnotations2020} dataset.
For ImageNet, we use the same set of 20 classes as in \cite{Wang2021} to have different labels than ID datasets.
Textures dataset was used both in \cite{Wang2021} and \cite{huangImportanceGradientsDetecting2021}.
Finally, TACO dataset was used in \cite{Zolfi2022} and other papers.
By this selection we should cover different class domains and provide new insights.

\paragraph{Classification model training}
We deploy two classifiers sourced from the study by Wang et al.\cite{Wang2021} featuring DenseNet-121 architecture. 
These classifiers were initially trained on ImageNet-1K and subsequently fine-tuned through the utilization of sigmoid function.
Data was augmented with random crops and flips. 
Achieved mAP is 87.51\% for PASCAL-VOC, 73.83\% for MS-COCO. 

\subsection{OOD detection}
\paragraph{JointEnergy, Mahalanobis}
We use the public implementation of original authors to produce OOD scores. 

\paragraph{GradNorm}
Since GradNorm has yet to be adopted for multi-label classifiers, it is necessary to adjust the original implementation. 
For our implementation, we have set the parameters in the same fashion as the authors of GradNorm did, with temperature T being equal to 1 and using the gradients weights of the last layer only. 
Although binary cross-entropy loss with logits is used for the multi-label classifier, it is still noteworthy that we compute cross-entropy loss here. 
The multi-label classifier does not utilize softmax on the network's output, but rather employs the sigmoid function to generate a K-long vector of class probabilities. 
This vector is exactly what we need to calculate GradNorm, and hence, we can utilize it in a way displayed below.

\noindent To generate GradNorm scores, one must execute the following steps:
\begin{itemize}
    \itemsep0em 
    \item use the classifier to produce the predicted probability distribution over classes $[K \times 1]$ vector
    \item divide the probability distribution by temperature T
    \item compute gradients of KL divergence as average the derivative of the cross-entropy loss for all labels
    \item perform backward pass of the network
    \item get the last layer's output and compute the first norm
\end{itemize}

\subsection{Evaluation settings}
\paragraph{Evaluation pipeline}
For every classifier, scores are computed from In-Distribution (ID) datasets as \textbf{inScores} 
and scores for Out-Of-Distribution (OOD) datasets as \textbf{outScores}. 
OOD detection is a binary classification so it requires the generation of ground truth labels. 
In the case of \textit{inScores}, the ground truth is set to 0 since we know that they contain labels which they were trained for. 
Alternatively, for \textit{outScores}, the ground truth is set to 1. 
Metrics are then computed from both \textit{inScores} and \textit{outScores} along with their corresponding ground truth labels.

\paragraph{Metrics}
The evaluation of performance is executed by means of metrics that are widely utilized in the field of Out-of-Distribution (OOD) detection. 
These metrics include:

\begin{enumerate}[label=(\alph*)]
\item FPR95 - the false positive rate of OOD samples when the true positive rate is at 95\%
\item AuROC - the area beneath the receiver operating characteristic curve
\item AuPR - the area beneath the precision-recall curve
\end{enumerate}

\subsection{Results}
In table bla you can see the quantitative results in \autoref{experiments:table1}.

\begin{table}[]
\begin{tabular}{llllllll}
\hline
                    & \multicolumn{1}{c}{}                         & \multicolumn{3}{c}{MS-COCO}                                                                & \multicolumn{3}{c}{PASCAL}                                          \\
                    &                                              & \multicolumn{1}{c}{\textbf{}} & \multicolumn{1}{c}{\textbf{FPR95}} & \textbf{AUROC}        & \textbf{AUPR}         & \textbf{}            & \textbf{}            \\
\textbf{OOD score} & \multicolumn{1}{c}{\textbf{detector family}} & \multicolumn{1}{c}{\textbf{}} & \multicolumn{1}{c}{\textbf{↓}}     & \multicolumn{1}{c}{↑} & \multicolumn{1}{c}{↑} & \multicolumn{1}{c}{} & \multicolumn{1}{c}{} \\ \hline
JointEnergy        & output-based                                 & 1                             & 2                                  & 3                     & 4                     & 56                   & 6                    \\
Mahalanobis        & feature-based                                & 1                             & 1                                  & 1                     & 1                     & 1                    & 1                    \\
GradNorm           & gradient-based                               & 1                             & 1                                  & 1                     & 1                     & 1                    & 1                    \\ \hline
\end{tabular}
\caption{Quantitative results for 3 Out-Of-Distribution (OOD) detector: JointEnergy, Mahalanobis and GradNorm. We want the minimal FPR95 and maximal AUROC and AUPR metrics.}
\label{experiments:table1}
\end{table}