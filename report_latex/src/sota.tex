\section{OOD detection in multi-class setting}
In 2015, Nguyen et al.~\cite{nguyenDeepNeuralNetworks2015} exploid the vulnerability of deep networks.
They found that one cause of overconfidence for OOD data is due to the nature of the fast-growing exponential function used in computing softmax probabilities.
Small changes to the logits can result in significant changes to the output distribution. The direct correlation between prediction probability and confidence in a softmax distribution is poor.

\subsection{MSP detector}
\paragraph{Maximum Softmax Probability (MSP) score}
In 2016, Hendrycks and colleagues~\cite{Hendrycks2016} took a significant step towards addressing the vulnerabilities of neural networks. The main findings can be concluded as follows. 
The probability of correctly classified examples is higher than that of misclassified and out-of-distribution examples. 
Thus, by capturing the statistics of prediction probabilities of correct examples, we can use them to detect whether an example is in error or abnormal. 
It is important to note that relying solely on softmax probabilities may lead to unreliable outcomes. 
However, analyzing the statistics of these probabilities is yet effective across various domains, including computer vision. 
This work has become a baseline for OOD detection in multi-class settings and is still used for benchmarking state-of-the-art methods.

\subsection{Out-of-DIstribution detector for Neural networks (ODIN)}
\paragraph{Temperature scaling and input pre-processing}
In 2017, Liang et al.~presented another output-based approach known as ODIN~\cite{Liang2017} 
(Out-of-DIstribution detector for Neural networks). 
ODIN is a technique that uses temperature scaling and input pre-processing (adding small perturbations). 
It aims to effectively separate the softmax probability distributions of in-and-out-of-distribution images. 
This separation enables more accurate and efficient OOD detection.
Temperature scaling involves scaling the logits (inputs to the softmax function) by a temperature parameter $T$ before computing the softmax function. 
This has the effect of sharpening the softmax output, making it easier to distinguish between in-distribution and out-of-distribution images.

\subsection{Mahalanobis distance detector}
\paragraph{Mahalanobis distance}
In 2018, Lee et al.~\cite{leeSimpleUnifiedFramework2018} introduced a new feature-based approach for detecting out-of-distribution (OOD) samples, which can be applied to any pre-trained softmax neural classifier without requiring to re-train.
The method uses a `generative' (distance-based) classifier to measure the probability density of test samples on feature spaces of Deep Neural Networks (DNNs). 
The authors assume that pre-trained low and upper level features can be fitted well by a class-conditional Gaussian distribution. 
Confidence score is defined using the Mahalanobis distance with respect to the closest class-conditional distribution. 

\subsection{Energy-based detector}
\paragraph{Energy score}
In 2021, Liu, Wang, et al.~\cite{liuEnergybasedOutofdistributionDetection2021} presented a new method that uses outputs of the DNNs. 
Instead of employing softmax scores, as with the MSP approach, they employ energy scores, 
which better distinguish between in- and out-of-distribution samples. 
Unlike softmax confidence scores, energy scores are theoretically aligned with the probability density 
of the inputs and are less susceptible to the overconfidence issue. 
Energy can be flexibly used as a scoring function for any pre-trained neural classifier, as well as a trainable cost function to explicitly shape the energy surface for OOD detection.
Energy score function was introduced in \autoref{energy} and for a given input 
$(\mathbf{x}, y)$ is defined as $E(\mathbf{x}, y) = -f_y (\mathbf{x})$. 
Without altering the parameterization of the neural network $f(\mathbf{x})$, we can define the \textbf{free energy function} $E(\mathbf{x}; f)$ over $\mathbf{x} \in \mathbb{R}^D$  in terms of the denominator of the softmax activation:

\begin{equation}
\centering
E(\mathbf{x}, f) = - T * \log{\sum_{i}^{K}{e^{f_i(\mathbf{x})/T}}}
\end{equation}

where K is the number of classes and $T$ is the temperature parameter.

\subsection{Fast Out-Of-Distribution Detector (FOOD)}
\paragraph{Statistical testing and additional output neuron}
In 2021, Amit and Levy~\cite{Amit2020} proposed a Fast Out-Of-Distribution Detector (FOOD) that efficiently detects out-of-distribution (OOD) samples with minimal inference time overhead. 
The paper presents a DNN model with a final Gaussian layer that models a density function for each class and a rapid OOD detector that does not require OOD samples for training or hyperparameter tuning. 
The paper uses a log likelihood ratio statistical test and an additional output neuron for OOD detection.

\subsection{GradNorm}
\paragraph{Categorizing OOD detectors}
We can categorize the OOD detection methods that have been implemented into two main groups: \textbf{output-based} and \textbf{feature-based}. 
Output-based detectors employ the output of neural networks which is then computed using an aggregation function (such as max or sum) and a scoring function to detect OOD instances. E
xamples of these methods include ODIN\cite{Liang2017}, Energy-based\cite{liuEnergybasedOutofdistributionDetection2021}, and MSP\cite{Hendrycks2016}. 
On the other hand, feature-based detectors utilize the feature space to distinguish between OOD and ID samples. 
Mahalanobis distance based detection\cite{Lee2018} is an example of this method.

\paragraph{Using KL divergence}
In 2021,~Huang, Rui and Geng~\cite{huangImportanceGradientsDetecting2021} introduced a novel approach for OOD detection called GradNorm. 
Their method utilizes information extracted from the \textbf{gradient space}. 
The main idea proposed in this paper is to use the vector norm of gradients, which are backpropagated from the \textbf{KL divergence} between the softmax output 
and a uniform probability distribution, as a means of detecting OOD inputs. 

Definition for KL divergence can be find in \autoref{KL:equation}.
For this method we set the reference distribution to be uniform $\mathbf{u} = [1/C, 1/C, ..., 1/C] \in \mathbb{R}^C$ . The predictive probability distribution is represented by the softmax output.
KL divergence then becomes:
\begin{equation}
\label{KL:equation2}
    D_{KL} = (\mathbf{u} \parallel \text{softmax}(f(\mathbf{x}))) =
    \frac{1}{C}\sum_{c=1}^{K}{\log{\frac{e^{f_c(\mathbf{x})/T}}
                                           {\sum_{j=1}^{C}{e^{f_j(\mathbf{x})/T}}}}} - H(\mathbf{u})
\end{equation}
where the first term is the cross-entropy loss with a uniform vector u, and the second term is a constant H(u). 
KL divergence measures how far the predictive distribution is from the uniform distribution. 
ID data is expected to have a larger KL divergence because the prediction tends to concentrate on one of the ground-truth classes and is therefore less uniformly distributed. 

\paragraph{GradNorm score}
This technique does not directly employ KL divergence, but instead utilizes the gradient vector norm, which is propagated backwards from the KL divergence. The OOD score is then defined as:
\begin{equation}
\label{KL:equation3}
    S(\mathbf{x}) = 
    \Vert
    \pdv{D_{KL}(\mathbf{u} \parallel \text{softmax}(f(\mathbf{x})))}
    {\mathbf{w}}
    \Vert
\end{equation}

where $\mathbf{w}$ is is the set of parameters in vector form and $\Vert$.$\Vert$ denotes $L_1$ norm\dots

At the end exploring gradient space proved to be useful and this become a new state-of-the-art method.